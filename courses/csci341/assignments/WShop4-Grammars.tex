\documentclass[11pt]{article}

\newcommand{\wshop}{
    4
}
\newcommand{\subtitle}{
    Grammars
}

% Page Setup
\usepackage{geometry}
\geometry{
    a4paper,
    margin={2.5cm}
}

% Basic Packages
\usepackage{amssymb}
\usepackage{stmaryrd}
\usepackage{amsmath}
\usepackage{amsthm}
\usepackage{mathtools}
\usepackage{mathpartir}
\usepackage{enumitem}
\usepackage{mathabx}

% Font
\usepackage{charter}

% Bibliography and index
\usepackage[backend=biber, style=numeric]{biblatex}
\addbibresource{refs.bib}
\usepackage{makeidx}
\makeindex

% Colors and Graphics
\usepackage[dvipsnames, x11names]{xcolor}
\usepackage{tikz}
\usetikzlibrary{
    cd,
    fit,
    calc,
    positioning,
    arrows,
    automata,
    shapes
}
\tikzset{
    baseline = (current bounding box.center),
    every state/.append style = {
        rectangle,
        rounded corners=5pt,
		inner sep = 3pt,
		minimum size = 18pt,
		initial text = {},
        fill=Azure1
	},
	every edge/.append style = {
		->,
		>=stealth,
		bend angle=10,
		thick
	}
}
\usepackage{musicography}
\usepackage{graphicx}
\usepackage{svg}
\graphicspath{../imgs/}

% Hyperlinks
\usepackage{hyperref}
\hypersetup{
    colorlinks,
    linkcolor   = black,
    filecolor   = RubineRed,
    urlcolor    = RubineRed,
    citecolor   = RubineRed,
    pdftitle    = {Notes on Behavioural PDEs}
}
\usepackage[capitalize]{cleveref}

% Environments
\theoremstyle{theorem} % In Italics
\newtheorem{theorem}                    {{\color{Purple}Theorem}}[section]
\newtheorem{lemma}          [theorem]   {{\color{Magenta}Lemma}}
\newtheorem{proposition}    [theorem]   {Proposition}
\newtheorem{corollary}      [theorem]   {Corollary}
\newtheorem{question}                   {{\color{red}Question}}

\theoremstyle{definition} % Not in italics
\newtheorem{definition}     [theorem]   {{\color{NavyBlue}Definition}}
\newtheorem{example}        [theorem]   {{\color{ForestGreen}Example}}
\newtheorem{problem}                    {{\color{BurntOrange}Problem}}

\theoremstyle{remark} % Subdued label
\newtheorem{remark}[theorem]        {{\color{Gray}Remark}}

% (1), (2), ...
\renewcommand\labelenumi{(\theenumi)}

% Go nuts with line breaks 
\allowdisplaybreaks

%%%%%%%%%%
% MACROS %
%%%%%%%%%%

\newcommand{\op}{\mathrm{op}}               % Opposite
\newcommand{\inv}{{-1}}                     % Inverse
\newcommand{\id}{\mathsf{id}}               % Identity f(x) = x
\newcommand{\Det}{\mathrm{Det}}             % determinize
\newcommand{\Lang}{\mathcal{L}}             % Language

\newcommand{\incl}{\mathsf{incl}}           % Inclusion
\newcommand{\proj}{\mathsf{proj}}           % Projection

% Numbers and Standard notation
\newcommand{\NN}{\mathbb{N}}                % 0, 1, 2, 3, 4, ...
\newcommand{\ZZ}{\mathbb{Z}}                % ..., -2, -1, 0, 1, 2, ...
\newcommand{\QQ}{\mathbb{Q}}                % n/m for n and m in \NN and m > 0
\newcommand{\RR}{\mathbb{R}}                % real numbers
\newcommand{\pRR}{\mathbb{R}_{+}}           % positive real numbers

\newcommand{\dom}{\mathrm{dom}}             % Domain
\newcommand{\cod}{\mathrm{cod}}             % Codomain

\newcommand{\Grph}{\operatorname{Grph}}     % Graph of a function

% Transitions
\newcommand{\tr}[1]{
    \mathrel{
        \raisebox{-1pt}{
            \(\xrightarrow{#1}\)
        }
    }
}
\newcommand{\bisim}{\mathrel{\raisebox{1pt}{\(\underline{\leftrightarrow}\)}}}

% Text
\newcommand{\code}[1]{\texttt{#1}}
\newcommand{\codeblock}[1]{
    \begin{center}
        \parbox{0.8\textwidth}{
            \ttfamily
            #1
        }
    \end{center}
}

% Boolean statements
\newcommand{\OR}{~\mathrm{or}~}
\newcommand{\AND}{~\mathrm{and}~}
\newcommand{\NOT}{\mathrm{not}~}
\newcommand{\IMPLIES}{~\mathrm{implies}~}
\newcommand{\FORALL}{\mathrm{for\ all}~}
\newcommand{\EXISTS}{\mathrm{there\ exists}~}
\newcommand{\SUCHTHAT}{~\mathrm{such\ that}~}



% Title
\title{CSCI 341 Workshop \wshop}
\author{\subtitle}
\date{
    \today
}

\pagestyle{empty}

\begin{document}


\maketitle

%%%%%%%%%%%%%%%%%%%%%%%%%%%%%%%%%%%%%%%%%%%%%%%%%%%%%%%%%%%%
% START OF WORK SHOP.                                      %
%%%%%%%%%%%%%%%%%%%%%%%%%%%%%%%%%%%%%%%%%%%%%%%%%%%%%%%%%%%%


\begin{problem}[Same Number]
    Show that the following language is not regular.
    \[
        L = \{w \in \{0,1\} \mid \text{\(w\) contains the same number of \(0\)s as it does \(1\)s}\}
    \] 
\end{problem}

\vspace{20em}

\begin{problem}[Linear Combination]
    Show that the following languages are not regular. 
    \begin{enumerate}
        \item \(L_1 = \{a^nb^nc^n \mid n \in \mathbb N \}\)
        
        \pagebreak
        
        \item \(L_2 = \{a^nb^{2n - 1} \mid n \in \mathbb N \}\)
        \vspace{25em}
        
        \item \(L_3 = \{a^{3n+1}b^{2n - 1} \mid n \in \mathbb N \}\)
        
        \pagebreak

    \end{enumerate}
\end{problem}

\begin{problem}[Less Than]
    Show that the language
    \[L_4 = \{a^nb^m \mid n,m \in \mathbb N \text{ and } n < m\}\]
    is not regular.
\end{problem}

\vspace{15em}

\begin{problem}[Balanced Parentheses]
    A string of parentheses, i.e., \(\mathtt{)}\) and \(\mathtt{(}\), is called <i>balanced</i> if every left-parenthese \(\mathtt{(}\) is eventually followed by a unique <i>matching</i> right-parenthese \(\mathtt{)}\). 
    For example, the following strings of parentheses are not balanced:
    \[
        \mathtt{(},
        \qquad \mathtt{))()},
        \qquad \mathtt{((())()}
    \]
    but the following strings of parentheses are:
    \[
        \varepsilon,
        \qquad \mathtt{()},
        \qquad \mathtt{(())()},
        \qquad \mathtt{((())())()}
    \]
    Let \(A = \{\mathtt{(}, \mathtt{)}\}\). 
    Prove that the language 
    \[
        L = \{w \in A^* \mid \text{\(w\) is balanced}\}
    \]
    is nonregular.
\end{problem}

\pagebreak

\begin{problem}[Palindromes]
    Now let \(A = \{0,1\}\) and recall that for any word \(w = a_1 a_2 \cdots a_n\), we define \(w^{\op} = a_n a_{n-1} \cdots a_2 a_1\).
    Consider the language below:
    \[
        L_{pal} = \{w \in A^* \mid w = ^{\op}\}
    \]
    The words in \(L_{pal}\) are precisely the \emph{palindromes}.
    Show that \(L_{pal}\) is not regular.
\end{problem}


\end{document}