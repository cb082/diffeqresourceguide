\documentclass[11pt]{article}

\newcommand{\pset}{
    3
}
\newcommand{\subtitle}{
    Determinization, 
    the Structure of \(\mathsf{Fin}\),
    Regular Expressions, and 
    Antimirov Derivatives
}
\newcommand{\duedate}{
    Friday, September 5
}

% Page Setup
\usepackage{geometry}
\geometry{
    a4paper,
    margin={2.5cm}
}

% Basic Packages
\usepackage{amssymb}
\usepackage{stmaryrd}
\usepackage{amsmath}
\usepackage{amsthm}
\usepackage{mathtools}
\usepackage{mathpartir}
\usepackage{enumitem}
\usepackage{mathabx}

% Font
\usepackage{charter}

% Bibliography and index
\usepackage[backend=biber, style=numeric]{biblatex}
\addbibresource{refs.bib}
\usepackage{makeidx}
\makeindex

% Colors and Graphics
\usepackage[dvipsnames, x11names]{xcolor}
\usepackage{tikz}
\usetikzlibrary{
    cd,
    fit,
    calc,
    positioning,
    arrows,
    automata,
    shapes
}
\tikzset{
    baseline = (current bounding box.center),
    every state/.append style = {
        rectangle,
        rounded corners=5pt,
		inner sep = 3pt,
		minimum size = 18pt,
		initial text = {},
        fill=Azure1
	},
	every edge/.append style = {
		->,
		>=stealth,
		bend angle=10,
		thick
	}
}
\usepackage{musicography}
\usepackage{graphicx}
\usepackage{svg}
\graphicspath{../imgs/}

% Hyperlinks
\usepackage{hyperref}
\hypersetup{
    colorlinks,
    linkcolor   = black,
    filecolor   = RubineRed,
    urlcolor    = RubineRed,
    citecolor   = RubineRed,
    pdftitle    = {}
}
\usepackage[capitalize]{cleveref}

% Environments
\theoremstyle{theorem} % In Italics
\newtheorem{theorem}                    {{\color{Purple}Theorem}}[section]
\newtheorem{lemma}          [theorem]   {{\color{Magenta}Lemma}}
\newtheorem{proposition}    [theorem]   {Proposition}
\newtheorem{corollary}      [theorem]   {Corollary}
\newtheorem{question}                   {{\color{red}Question}}

\theoremstyle{definition} % Not in italics
\newtheorem{definition}     [theorem]   {{\color{NavyBlue}Definition}}
\newtheorem{example}        [theorem]   {{\color{ForestGreen}Example}}
\newtheorem{problem}                    {{\color{BurntOrange}Problem}}

\theoremstyle{remark} % Subdued label
\newtheorem{remark}[theorem]        {{\color{Gray}Remark}}

% (1), (2), ...
\renewcommand\labelenumi{(\theenumi)}

% Go nuts with line breaks 
\allowdisplaybreaks

%%%%%%%%%%
% MACROS %
%%%%%%%%%%

\newcommand{\op}{\mathrm{op}}               % Opposite
\newcommand{\inv}{{-1}}                     % Inverse
\newcommand{\id}{\mathsf{id}}               % Identity f(x) = x
\newcommand{\Det}{\mathrm{Det}}             % determinize
\newcommand{\Lang}{\mathcal{L}}             % Language

\newcommand{\incl}{\mathsf{incl}}           % Inclusion
\newcommand{\proj}{\mathsf{proj}}           % Projection

% Numbers and Standard notation
\newcommand{\NN}{\mathbb{N}}                % 0, 1, 2, 3, 4, ...
\newcommand{\ZZ}{\mathbb{Z}}                % ..., -2, -1, 0, 1, 2, ...
\newcommand{\QQ}{\mathbb{Q}}                % n/m for n and m in \NN and m > 0
\newcommand{\RR}{\mathbb{R}}                % real numbers
\newcommand{\pRR}{\mathbb{R}_{+}}           % positive real numbers

\newcommand{\dom}{\mathrm{dom}}             % Domain
\newcommand{\cod}{\mathrm{cod}}             % Codomain

\newcommand{\Grph}{\operatorname{Grph}}     % Graph of a function

% Transitions
\newcommand{\tr}[1]{
    \mathrel{
        \raisebox{-1pt}{
            \(\xrightarrow{#1}\)
        }
    }
}
\newcommand{\bisim}{\mathrel{\raisebox{1pt}{\(\underline{\leftrightarrow}\)}}}

% Text
\newcommand{\code}[1]{\texttt{#1}}
\newcommand{\codeblock}[1]{
    \begin{center}
        \parbox{0.8\textwidth}{
            \ttfamily
            #1
        }
    \end{center}
}

% Boolean statements
\newcommand{\OR}{~\mathrm{or}~}
\newcommand{\AND}{~\mathrm{and}~}
\newcommand{\NOT}{\mathrm{not}~}
\newcommand{\IMPLIES}{~\mathrm{implies}~}
\newcommand{\FORALL}{\mathrm{for\ all}~}
\newcommand{\EXISTS}{\mathrm{there\ exists}~}
\newcommand{\SUCHTHAT}{~\mathrm{such\ that}~}



% Title
\title{CSCI 341 Problem Set \pset}
\author{\subtitle}
\date{Due
    \duedate
}

\begin{document}

\maketitle

Don't forget to check the webspace for hints and additional context for each problem!

\subsection*{Determinization}

\begin{problem}
    [Determinizing is Deterministic]
    Prove that for any automaton \(\mathcal A = (Q, A, \delta, F)\), \(\mathrm{Det}(\mathcal A)\) is total deterministic.    
\end{problem}

\begin{proof}[Solution.]
    
\end{proof}

\begin{problem}
    [Determinized State, Completing the Proof]
    Let \(\mathcal A = (Q, A, \delta, F)\) be an automaton, and let \(\mathrm{Det}(\mathcal A)\) be its determinization.
    Prove that for any state \(x \in Q\), 
    \[
        \mathcal L(\mathcal A, x) \supseteq \mathcal L(\mathrm{Det}(\mathcal A), \{x\})
    \]    
\end{problem}

\begin{proof}[Solution.]
    
\end{proof}

\begin{problem}
    [You Got Options]
    Find the smallest automaton (not necessarily total or deterministic) with a state that accepts the language 
    \[
        L = \{ab^n \mid n \in \mathbb N\} \cup \{ac^n \mid n \in \mathbb N\} \cup \{a(bc)^n \mid n \in \mathbb N\}
    \]
    over the alphabet \(A = \{a,b,c\}\).
    Use determinization to find a deterministic automaton with a state that accepts the same language.  
\end{problem}

\begin{proof}[Solution.]
    
\end{proof}

\subsection*{the Structure of \(\mathsf{Fin}\)}

\begin{problem}
    [Finish Closed under Complement]
    Let \(\mathcal A' = (Q, A, \delta, Q\setminus F)\).
    Prove that \(\mathcal L(\mathcal A', x) = A^* \setminus L\).
\end{problem}

\begin{proof}[Solution.]
    
\end{proof}

\begin{problem}
    [Intersection-product Construction]
    Let \(\mathcal A_1 = (Q_1, A, \delta_1, F_1)\) and \(\mathcal A_2 = (Q_2, A, \delta_2, F_2)\) be total deterministic automata, and let \(x \in Q_1\) and \(y \in Q_2\).
    Let \(L_1 = \mathcal L(\mathcal A_1, x)\) and \(L_2 = \mathcal L(\mathcal A_2, x)\).
    Change the accepting states in the union-product construction to obtain an automaton \(\mathcal A_1 \otimes \mathcal A_2 = (Q_1 \times Q_2, A, \delta^\times, F^
    \otimes)\) such that \(\mathcal L(\mathcal A_1 \otimes \mathcal A_2, (x, y)) = L_1 \cap L_2\).
    Explain how to obtain a proof of the Closure under Intersection Theorem from the proof of the Closure under Union Theorem (i.e., what would you change?).
\end{problem}

\begin{proof}[Solution.]
    
\end{proof}

\subsection*{Regular Expressions}

\begin{problem}
    [Intersections and Complements]
    Show that the following two languages are regular over \(A = \{a, b\}\). 
    \begin{enumerate}
        \item \(L_6 = \mathcal L(b^*a(a + b)^*) \cap \mathcal L(a^*b(a + b)^*)\)
        \item \(L_7 = A^* \setminus L_6\)
    \end{enumerate}
\end{problem}

\begin{proof}[Solution.]
    
\end{proof}

\subsection*{Antimirov Derivatives}

\begin{problem}
    [Some Nested Derivatives]
    Consider the regular expression \(r = (a(b+c^*) + b)^*\) over the alphabet \(A = \{a,b,c\}\).
    \begin{enumerate}
        \item Name three different words \(w_0,w_1,w_1 \in A^*\) that are not in \(\mathcal L(r)\), i.e., \(w_0,w_1,w_2 \notin \mathcal L(r)\).
        <li>Use the inequalities in the proof of the Linear Bound on Antimirov Derivatives Lemma to determine an upper bound on the number of states in the automaton \(\langle r\rangle_{\mathcal A_{Ant}}\) generated by \(r\) in \(\mathcal A_{Ant}\), i.e., \(\#(r)\).</li>
        \item Now draw a state diagram of \(\langle r\rangle_{\mathcal A_{Ant}}\).
        \item How many formation rules were used to form the regular expression \(r\)? 
            How does this number of formation rules compare to the number of states in \(\langle r\rangle_{\mathcal A_{Ant}}\)?
    \end{enumerate}
\end{problem}

\begin{proof}[Solution.]
    
\end{proof}

\end{document}